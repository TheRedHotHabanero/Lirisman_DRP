\documentclass[12pt]{article}
\usepackage{amsmath, amssymb, amsthm}
\usepackage{setspace}
\usepackage{graphicx}
\usepackage{mathtools}
\usepackage{float}
\usepackage{algorithm}
\usepackage{algpseudocode}
\usepackage{listings}
\usepackage{enumitem}
\usepackage[russian]{babel} 

\usepackage[margin=1in]{geometry}

\begin{document}

\title{\textbf{Марковская цепь как случайный процесс}}
\author{Выполнила Карина Лирисман, группа Б01-008}
\date{}

\maketitle

\section{Введение}

Марковский процесс - это стохастический процесс, в котором вероятность перехода состояний зависит только от текущего состояния и не зависит от предыдущих состояний (т.е. обладает свойством Маркова). Марковский процесс удобен для моделирования многих систем, таких как финансовые рынки, экономические и социологические процессы, погода и другие случайные явления.

\subsection{Марковская цепь}

Марковская цепь - это стохастический процесс, который состоит из состояний и переходов между этими состояниями. 

Формально, мы можем определить марковскую цепь $\{X_n\}$ на конечном множестве состояний $S = \{s_1, s_2, ..., s_n\}$ следующим образом. Если $X_n$ находится в состоянии $s_i$, то вероятность перейти в состояние $s_j$ на следующем шаге (т.е. в момент времени $n+1$) обозначим $P_{ij}$, где $i,j \in S$ и $\sum_{j}P_{ij}=1$.

Марковская цепь является временно однородной, если матрица переходных вероятностей $P_{ij}$ не меняется со временем. Если матрица переходных вероятностей зависит от времени, то мы говорим, что марковская цепь нестационарна.

\section{Случайный процесс}

Случайный процесс - это семейство случайных переменных $\{X_t : t \in T\}$ на общем вероятностном пространстве. Для каждого фиксированного $t \in T$, $X_t$ - это случайная переменная. 

Случайный процесс также можно рассматривать как индексированный набор случайных величин в некоторый момент времени. Если множество временных меток $T$ является интервалом времени, то случайный процесс называется непрерывным во времени. Если $T$ конечен или состоит из дискретных меток времени, то случайный процесс называется дискретным во времени.

В данном случае переходы системы из одного в другое состояние возможно в строго определенные моменты времени ${t_0}$, ${t_1}$, …, ${t_n}$ , а случайный
процесс $X(t_i)$ в промежутках между указанными моментами времени сохраняет свое состояние (см рис.\ref{figure1}). Такие процессы встречаются при машинной обработке информации в цифровых ЭВМ. 

\begin{figure}[!ht]
\begin{center}
\includegraphics[scale=0.5]{1_pic.png}\caption{}\label{figure1}
\end{center}
\end{figure}

Случайный процесс называется дискретным Марковским процессом, если его состояния можно пронумеровать и переход из одного в другое состояния происходит скачком (см рис.\ref{figure2})

\begin{figure}[!ht]
\begin{center}
\includegraphics[scale=0.5]{2_pic.png}\caption{}\label{figure2}
\end{center}
\end{figure}

В этом случае случайный процесс $X(t)$ принимает дискретные значения $х_i$, i = 1 … n, время t изменяется непрерывно.

\section{Теорема Колмогорова}

Теорема Колмогорова утверждает, что для любой временно однородной марковской цепи с конечным числом состояний существует минимальное неотрицательное целое число n, такое что $P_{ij}^{(n)} > 0$ для всех $i,j \in S$. То есть, начиная с некоторого момента времени, вероятность перехода между любой парой состояний становится положительной.

Теорема Колмогорова устанавливает свойство стационарности марковских цепей. Она гласит, что если марковская цепь удовлетворяет условию эргодичности, то существует единственное стационарное распределение вероятностей в этой цепи, которое не зависит от начального состояния.

\section{Цепь Маркова как случайный процесс}

Марковская цепь – это последовательность случайных событий, где вероятность следующего события зависит только от текущего. В этом смысле марковские цепи являются случайным процессом. В отличие от других случайных процессов, марковские цепи не имеют памяти, то есть не сохраняют информацию о предыдущих событиях.

Особый интерес представляют марковские процессы с конечным числом состояний, которые называются цепями Маркова с дискретным временем. Для таких процессов можно вычислить стационарные распределения и вероятности переходов между состояниями.

В целом, марковские процессы представляют собой мощный инструмент для моделирования различных систем, в которых нужно учитывать зависимости только от текущего состояния. Их применение позволяет прогнозировать изменения в системе и оптимизировать различные параметры для достижения желаемых результатов.

Марковские цепи находят применение в различных областях, таких как физика, экономика, биология и т.д. Они позволяют моделировать различные системы и прогнозировать их поведение в будущем. Также они являются основой для различных алгоритмов машинного обучения, таких как скрытые марковские модели

\subsection{Марковский случайный процесс с дискретными состояниями и дискретным временем (цепь Маркова)}

Дискретным Марковским процессом с дискретными состояниями и дискретным временем называют процесс, в котором возможные состояния системы S1, S2, … , Sn можно перечислить (пронумеровать), а сам процесс перехода системы из одного в другое состояние происходит в фиксированные моменты времени $t_1$, $t_2$, $t_3$, … $t_n$. В промежутках между этими моментами времени система сохраняет свое состояние.

При любом шаге перехода системы события переходов образуют полную группу и несовместны, так как система может находиться лишь в какомто одном из состояний, а не в двух и более сразу. При этом рассматриваются
все возможные состояния системы. 

Если каждое состояние системы в k-й момент времени свяжем с вероятностью, то получим вероятности состояний $P(k)_1$, $P(k)_2$, $P(k)_3$, … , $P(k)_n$. Для
любого момента времени $P(k)_1$ + $P(k)_2$ +,…, + $P(k)_n$ = 1

Для любого момента времени $t_1$, $t_2$, $t_3$,…, $t_k$, существуют вероятности перехода системы из любого состояния в любое другое $P_i$ $j(k)$ (i, j = 1, 2, , n). Их
называют вероятностями перехода или переходными вероятностями. Эти вероятности отличны от нуля, если система переходит из одного состояния в
другое, и равны нулю, если на данном шаге система остается в прежнем состоянии. 

Если вероятности перехода не зависят от номера шага (момента време-
ни) и не изменяются от шага к шагу, то такой Марковский процесс называют
однородным. В противном случае Марковский процесс называют неоднородным.

Вероятностная картина возможных состояний системы и ее переходов
может быть задана матрицей Р, элементами которой являются переходные
вероятности: 

\begin{center}

$ P(k)= \begin{bmatrix}
		\ p(k)_{11}& \ p(k)_{12}...& \ p(k)_{1n}\\
		\ p(k)_{21}& \ p(k)_{22}...& \ p(k)_{2n}\\
		\ p(k)_{n1}& \ p(k)_{n2}...& \ p(k)_{nn}\\
		\end{bmatrix}
$
\end{center}

Матрица переходных вероятностей обладает следующими свойствами:
- сумма вероятностей, стоящих в каждой строке матрицы равна единице;
- на главной диагонали матрицы стоят вероятности того, что система не
выйдет из состояния $S_i$ а останется в нем;
- если переходная вероятность $P(k)_ij$ = 0, то это означает, что на данном
шаге система не может перейти из состояния $S_i$, в состояние $S_j$.

Имея размеченный граф состояний или матрицу переходных вероятностей и, зная начальное состояние системы, можно найти все возможные вероятности состояний системы для любого момента времени.

Для этого используют следующие уравнения:

- для однородного Марковского процесса:
\begin{center}
$p(k)_j = \sum_{i=1}^n p_i (k-1)  p_{ij} , j = 1, 2, ... , n$
\end{center}

- для неоднородного Марковского процесса:
\begin{center}
$p(k)_j = \sum_{i=1}^n p_i (k-1)  p(k)_{ij} , j = 1, 2, ... , n$
\end{center}

\section{Литература}

\begin{enumerate}

\item Ширяев А. Н. Ш64 Вероятность. В 2-х кн. –– 3-е изд., перераб. и доп. ––М.: МЦНМО, 2004. ISBN 5-94057-036-4 Кн. 2. –– 408 с. –– ISBN 5-94057-106-9

\item Булинский А. В., Ширяев А. Н. Теория случайных процессов. — М.: ФИЗМАТЛИТ, 2005. - 408 с. - ISBN 5-9221-0335-0.

\item Статистика случайных процессов (нелинейная фильтрация и смежные вопросы), Лицер Р.Ш., Ширяев А.Н., Главная редакция физико-математической литературы изд-ва "Наука" 1974.

\end{enumerate}

\end{document}